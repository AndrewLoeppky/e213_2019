\documentclass{article}

\usepackage{geometry}
\usepackage{multicol}
\usepackage{fancyhdr}
\usepackage{lastpage}
\geometry{letterpaper,top=50pt,hmargin={20mm,20mm},headheight=15pt}
\usepackage{verbatim}
\usepackage{booktabs}
\usepackage{pgfplots}
\usepackage{lastpage}
\usepackage{tikz}
\usetikzlibrary{shapes,arrows}
\definecolor{dark-green}{rgb}{0.0, 0.5, 0.0}


%%%<
%\usepackage[active,tightpage]{preview}
%%\PreviewEnvironment{tikzpicture}
%\setlength\PreviewBorder{5pt}%
%%%>
\pagestyle{fancy}

\fancypagestyle{first}{
\lhead{Quiz 4}
\chead{E213 - Thur, March 28, 2019}
\rhead{page~\thepage/\pageref{LastPage}}
\lfoot{} 
\cfoot{} 
\rfoot{}
}


\begin{document}
\pagestyle{first}


\large{EOSC 213 Quiz 4} \hspace{10cm} \large{March 28$^{\textrm{th}}$, 2019}

\large{Name:} \hspace{12cm} \large{ID: }
\begin{center}
\Huge{EOSC 213 - Quiz}
\end{center}

\rule{\textwidth}{1pt}

\large{\textbf{Instructions (30 points in total)}}
\begin{multicols}{2}
\begin{itemize}
\item Read the examination before beginning.
\item Calculators are allowed (if you don't have one, just give the expression to type in a calculator).
\item You have exactly 45 minutes for the examination.
\item Be as precise and clear as possible.
\item This is a closed book examination.
\item If you get stuck, make an assumption, state what it is and try to carry on.
\end{itemize} 
\end{multicols}



\rule{\textwidth}{1pt}

\textbf{Question 0: English is a funny language test}

\begin{description}
\item [Q0] Complete the sentence with your favourite word. \textit{If vegetarians consume vegetables, then humanitarians consume .....}  \textbf{[1 point]}
\vspace{0.5cm}

\end{description}

\textbf{Question 1: Finite-difference approximations}

Let us consider the exponential function described in equation \ref{eq:exp}

\begin{equation}
f(x) = \mathrm{exp}(x). \label{eq:exp}
\end{equation} 


\begin{description}
\item [Q1a] Compute a first-order approximation to the first derivative of the exponential function at $x = 0$ ($f'(0)$) using $\Delta x  = 0.2$.  \textbf{[2 points]}
\vspace{3cm}
\item [Q1b] Compute a second-order approximation to the first derivative of the exponential function at $x = 0$ ($f'(0)$) using $\Delta x  = 0.2$.  \textbf{[2 points]}
\vspace{3cm}

\end{description}




\textbf{Question 2: Diffusion Boundary Value Problem}

The 1D transient diffusion equation with homogeneous diffusion coefficient can be written as:
\begin{equation}
\frac{\partial c}{\partial t} = D \frac{\partial^2 c}{\partial x^2}. \label{eq:diff}
\end{equation} Let us consider the physical domain $ x \in \left[ -1 ; 1 \right] \mathrm{m}$, with the specified (Dirichlet) boundary conditions $c(1,t) = c(-1,t) = 0$ (where, for example, $c(1,t)$ should be read as the concentration at $x=1$ and all times $t$) and initial condition $c(x, 0) = c_0 \, \mathrm{cos}\left(\frac{\pi x}{2} \right)$. Consider the following function:
\begin{equation}
c(x,t) = c_0 \, \mathrm{cos}\left(\frac{\pi x}{2} \right) \mathrm{exp}\left(-\alpha t \right)  \label{eq:sol}
\end{equation}



\begin{description}
\item [Q2a] Show that the function described in equation \ref{eq:sol} satisfies the two boundary conditions at all times \textbf{[2 points]}
\vspace{2cm}

\item [Q2b] Show that the function described in equation \ref{eq:sol} satisfies the partial differential equation \ref{eq:diff} at every point in the domain. \textbf{Justify} your answer. \textbf{[2 points]}

\vspace{3cm}





\item [Q2c] Concentrations were measured at four different times and are represented in the graph below. The messy person who did the measurements does not remember at which time these measurements were taken. Can you help him, using your physical intuition? Label the 4 curves in termporal order where $t_1< t_2<t_3<t_4$ .  \textbf{[2 points]}
\end{description}

\begin{tikzpicture}
\begin{axis}[width = 8cm, xmin=-1,xmax = 1, ymin = 0, ymax = 1, height = 6cm, grid = both, xlabel = {x-axis}, ylabel = {$c$}, legend pos = outer north east]
\addplot[red, mark = o, mark size = 2 pt, line width = 0.1pt] table [x=x, y=c1]{diff.txt}; \addlegendentry{(b)}
\addplot[black, mark = square, mark size = 2pt, line width = 0.1 pt] table [x=x, y=c3]{diff.txt}; \addlegendentry{(a)}
\addplot[dark-green, mark = triangle, mark size = 2 pt, line width = 0.1pt] table [x=x, y=c2]{diff.txt}; \addlegendentry{(c)}
\addplot[blue, mark = diamond, mark size = 2pt] table [x=x, y=c4]{diff.txt}; \addlegendentry{(d)}
\end{axis}
\end{tikzpicture}

\begin{description}
\item [Q2d] What can you say about the change in the total mass in the system with time? Is that consistent with the boundary conditions? Explain. \textbf{[2 points]}
\vspace{2cm}

\item [Q2e] Use your physical reasoning (and equation \ref{eq:sol}) to describe the asymptotic/final solution (concentration versus $x$).. \textbf{[2 points]}
\vspace{2cm}

\end{description}

\textbf{Question 3: Conservation equation}
The general conservation (or sometimes called continuity) equation says:
\begin{equation}
\frac{\partial \, \mathrm{stuff}}{\partial t} = - \overrightarrow{\nabla } \cdot\overrightarrow{j},
\end{equation} or, in cartesian coordinates
\begin{equation}
\frac{\partial \, \mathrm{stuff}}{\partial t} = - \frac{\partial j_x}{\partial x} - \frac{\partial j_y}{\partial y} - \frac{\partial j_z}{\partial z}.
\end{equation} $ \overrightarrow{j} $ is a flux vector that describes the rate at which "stuff", the conserved quantity, fluxes (moves).


\begin{description}
\item [Q3a] Provide two examples of what "stuff", the \textbf{conserved quantity}, could represent. \textbf{[2 points]}
\vspace{2.5cm}

\item [Q3b] Write the PDE conservation law (either in nabla notation or in cartesian coordinates) for the case of the diffusion of a solute (not in a porous media) where the flux is given by Fick's law $ \overrightarrow{j_{\mathrm{diff}}} = - D \overrightarrow{\nabla} c$ \textbf{[3 points]}
\vspace{2.5cm}


\item [Q3c] What is the physical meaning of the term $\frac{\partial \, \mathrm{stuff}}{\partial t}$ in the case of a diffusing solute? Provide a one or two sentence(s) explanation. \textbf{[2 points]}
\vspace{2.5cm}

\item [Q3d] Write the PDE conservation law (either in nabla notation or in cartesian coordinates) for the case where the flux is given by advection $ \overrightarrow{j_{\mathrm{adv}}} = \overrightarrow{v} c$ \textbf{[3 points]}
\vspace{2.5cm}

\end{description}

\newpage
\textbf{Question 4: Python}
\begin{description}
\item[Q4a 3 pts] Consider the following dataframe \textbf{df}, with rows, columns and index given by:

\begin{tabular}{lrrrr}
\toprule
{} &  temperature &      veloc &      mass &  mass\_tenths \\
date                      &              &            &           &              \\
\midrule
2019-03-25 05:00:00+00:00 &   267.439863 &  74.146461 &  0.206719 &          0.2 \\
2019-03-25 05:10:00+00:00 &   281.372218 &  -2.317762 &  0.611744 &          0.6 \\
2019-03-25 05:20:00+00:00 &   278.318157 &   3.683598 &  0.296801 &          0.3 \\
2019-03-25 05:30:00+00:00 &   266.754425 & -83.851746 &  0.738440 &          0.7 \\
2019-03-25 05:40:00+00:00 &   271.826184 & -68.338026 &  0.879937 &          0.9 \\
\bottomrule
\end{tabular}

In the space below write python statements (single lines) that would
return the following:

\begin{enumerate}
\item The row corresponding to March 25, 2019 at 5:30 am UCT
\item The temperature for March 25, 2019 at 5:10 am UCT
\item All temperatures in the dataframe
\end{enumerate}

\vspace{5cm}

\item[Q4b 3pts] Write a python snippet that would use groupby to find
  the median velocity for all objects with \verb+mass_tenths+=0.7 kg for this
  dataframe
  
\end{description}







\end{document}
%%% Local Variables:
%%% mode: latex
%%% TeX-master: t
%%% End:
